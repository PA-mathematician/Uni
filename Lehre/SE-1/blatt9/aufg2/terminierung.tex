\documentclass{article}

\usepackage[utf8x]{inputenc}
\usepackage[ngerman]{babel}
\usepackage{amsmath,amsfonts,amssymb}

\DeclareMathOperator{\length}{length}
\DeclareMathOperator{\tail}{tail}

\usepackage{fancyvrb}
\DefineVerbatimEnvironment{code}{Verbatim}{fontsize=\small}

\author{Manuel Hoffmann}
\title{Induktion}

\begin{document}

\section{merge}
\begin{code}
  merge :: ([Integer], [Integer]) -> [Integer]
  merge ([], b) = b
  merge (a, []) = a
  merge (a:a1, b:b1) = if a < b then
                       a : merge (a1, b:b1)
                  else b : merge (a:a1, b1)
\end{code}
Die Funktion ist definiert auf einem Tupel aus Integer-Listen, das heißt, wir wissen auf jeden Fall

\begin{equation*}
P \subseteq \text{[Integer]} \times \text{[Integer]}
\end{equation*}
eine größere Parametermenge verbietet uns das Typsystem.

Nach Aufgabenstellung können wir uns auf endliche Listen beschränken, d.h
\begin{gather*}
  P \subseteq \text{[Integer]} \times \text{[Integer]}, \text{ mit [Integer] endlich}
  \intertext{Oder auch:}
  P \subseteq \{(a,b) | a, b \in \text{[Integer]} \land a, b \text{ endlich} \}
\end{gather*}

Wenn wir uns die rekursiven Aufrufe ansehen, erkennen wir, dass immer ein e Liste der beiden verkürzt wird, daher bietet sich als Deltafunktion die gesammte Länge, d.h. die Summe der beiden Längen an. Die Noethersche Ordnung wäre damit $(\mathbb{N}, \le)$
Außerdem sehen wir, dass wir den Parameterbereich nicht weiter einschränken brauchen, somit haben wir jetzt:

\begin{gather*}
N = (\mathbb{N}, le)\\
P = \{(a,b) | a, b \in \text{[Integer]} \land a, b \text{ endlich} \}\\
\\
\delta : P \rightarrow N \\
\delta (a, b) = \length a + \length b
\end{gather*}

Um mit dieser $\delta$-Funktion jetzt etwas untersuchen zu können, sollten wir aber genauer aufschreiben, wie die Funktionen $G_1$ und $G_2$, die die Veränderung des formalen Parameters für den rekursiven Aufruf beschreibt, aussieht:

\begin{gather*}
  G_1 : P \rightarrow P\\
  G_1(x,y) = (\tail x, y)
\end{gather*}
\begin{gather*}
  G_2 : P \rightarrow P\\
  G_2(x,y) = (x, \tail y)
\end{gather*}

Jetzt müssen wir \emph{erstens} Prüfen, dass die rekursiven aufrufe wieder auf zulässigen Parametern erfolgen:

Da die Listen endlich lange sind und im Falle eines rekursiven Aufrufs mindestens ein Element haben, und der rekursiven Aufruf jeweils nur um ein $\tail$ verändert ist, bleiben wir in $P$.

\end{document}