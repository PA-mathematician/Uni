\documentclass{article}

\usepackage[utf8x]{inputenc}
\usepackage[ngerman]{babel}
\usepackage{amsmath,amsfonts,amssymb}

\newcommand*{\thecheckbox}{\hss[\hss]} 
\newenvironment*{checklist} 
{\list{}{\itemsep -2pt 
\renewcommand*{\makelabel}[1]{\thecheckbox}}} 
{\endlist} 


\author{Manuel Hoffmann}
\title{SE1-Checkliste}

\begin{document}

Diese Aufzählung erhebt nicht den Anspruch, vollständig zu sein. Es soll euch nur eine Hilfe sein, damit ihr auch alles lernt. Wenn ihr fragen habt (z.B. weil ich etwas komisch formuliert habe), mailt mir an m_hoffmann09@cs.uni-kl.de - frohes Lernen!

\section{Blatt 1}

\begin{checklist}
  \item Weiß ich, wie sich Software zu anderen technischen Systemen verhält? 
  \item Kenne ich die "`Dimensionen"' von Softwareentwicklung?
\end{checklist}

\section{Blatt 2}

\begin{checklist}
  \item Kann ich einen informell beschriebenen Vorgang auch als expliziten Algorithmus beschreiben?
  \item \emph{Haskell:} Kann ich Funktionenen deklarieren?
  \item \emph{Haskell:} Kann ich Funktionen aufrufen?
  \item Kenne ich die Definition für eine kontextfreie Grammatik?
  \item Weiß ich, was der Unterschied zwischen Syntax und Semantik ist?
\end{checklist}

Freiwillige Zusatzaufgabe: 2e

\section{Blatt 3}

\begin{checklist}
  \item Kann ich Syntaxdiagramme lesen?
  \item Kann ich Syntaxdiagramme in Produktionsregeln einer kontextfreien Sprache übersetzen?
  \item Kann ich aus einer kontextfreien Sprache Sätze bilden?
  \item Kann ich herausfinden, ob einer Satz von einer bestimmten kontextfreien Sprache erzeugt wird?
  \item Weiß ich, was eine "`Linksableitung"' ist?
  \item Kann ich zeigen, dass eine Grammatik eindeutig ist?
  \item Kann ich anhand einer informellen Anforderung (Text) eine kontextfreie Grammatik angeben?
  \item Kenne ich den unterschied zwischen Sprache und Grammatik?
  \item Kann ich eine Sprache in Mengennotation angeben?
\end{checklist}

Freiwillige Zusatzaufgabe: 2d

\section{Blatt 4}

\begin{checklist}
  \item \emph{Haskell:} Kenne ich alle elementare Typen?
  \item \emph{Haskell:} Weiß ich, wo ich Klammern setzen muss und wo nicht?
  \item \emph{Haskell:} Erkennte ich den Typ verschiedener Ausdrücke?
  \item \emph{Haskell:} Weiß ich, welche Ausdrücke welche Fehler nach sich ziehen?
  \item \emph{Haskell:} Weiß ich, was eine Abstraktion ist?
  \item \emph{Haskell:} Weiß ich, wie ich Funktionen rekursiv definiere?
\end{checklist}

Freiwillige Zusatzaufgaben: 7d, 7e, 7f, 7g

\section{Blatt 5}

\begin{checklist}
  \item \emph{Haskell:} Kann ich Fallunterscheidungen mit Pattern-Matching realisieren?
  \item \emph{Haskell:} Kann ich Fallunterscheidungen mit Guards realisieren?
  \item \emph{Haskell:} Kann ich eigene Datenstrukturen bzw. Typen angeben?
  \item \emph{Haskell:} Kann ich bei meinen eigenen Datenstrukturen verhalten erzwingen?
  \item \emph{Haskell:} Weiß ich, wie ich mit Konstruktoren verschiedene Ausprägungen meiner eigenen Datenstrukturen unterscheiden kann?
  \item \emph{Haskell:} Kann ich Diskriminator- und Selektorfunktionen schreiben?
  \item \emph{Haskell:} Kenne ich die beiden wichtigsten rekursiven Datenstrukturen?
  \item \emph{Haskell:} Kann ich in rekursiven Datenstrukturen navigieren?
\end{checklist}

\section{Blatt 6}

\begin{checklist}
  \item \emph{Haskell:} Kann ich zu einer gegebene Grammatik Worte von einem Programm erzeugen lassen?
  \item \emph{Haskell:} Kann ich von einem Programm erkennen lassen, ob ein gegebenes Wort von einer gegebenen Grammatik erzeugt werden kann?
  \item Kenne ich die Prinzipien von Selectionsort, Insertionsort, Mergesort, Quicksort und Bubbelsort?
  \item \emph{Haskell:} Kann ich Sortieralgorithmen in Haskell beschreiben?
  \item \emph{Haskell:} Erkenne ich, wie oft gewisse Operationen in Abhängigkeit der Eingabegröße ausgeführt werden?
  \item \emph{Haskell:} Kann ich Funktionen verketten?
\end{checklist}

Freiwillige Zusatzaufgaben: 1a, 1b, 6a, 6b, 6c

\section{Blatt 7}

\begin{checklist}
  \item \emph{Haskell:} Kann ich funktionale Anforderungen an parametrische Typen stellen?
  \item \emph{Haskell:} Kann ich zu einem Ausdruck den allgemeinsten Typ angeben?
  \item \emph{Haskell:} Kann ich zu einem allgemeinen Typ ein Funktionsbeispiel angeben?
  \item \emph{Haskell:} Kann ich mit Funktionen höherer Ordnung umgehen?
  \item \emph{Haskell:} Kenne ich die Definitionen von foldr, foldl, map, reduce?
  \item \emph{Haskell:} Kann ich mit Hilfe diesen Funktionen andere Funktionen kürzer nachbauen?
  \item Kenne ich den Unterschied zwischen strikt und nicht-strikt?
  \item Kenne ich den Unterschied zwischen aktuellem und formalem Parameter?e
  \item Kenne ich die Auswertungsstrategie von Haskell?
  \item 
\end{checklist}

Freiwillige Zusatzaufgaben: 4i, 4j, 4k, 4l, 5b, 8a, 8b, 8c, 8d, 8e

\section{Blatt 8}

\begin{checklist}
  \item Weiß ich, was ich mit einer Parameterinduktion beweisen kann?
  \item Kann ich das mit einer Parameterinduktion zeigen?
  \item Kann ich zeigen, dass etwas eine Noethersche Ordnung ist?
  \item Kann ich zu einer gegebenen rekursiven Funktion in Haskell eine Deltafunktion angeben, mit der ein Terminierungsbeweis geführt werden kann?
  \item Weiß ich, was zu einem Terminierungsbeweis gehört?
  \item Kann ich eine möglichst große Parametermenge finden, für die eine Funktion terminiert?
  \item \emph{Java:} (Nachdem wir jetzt auch Objektorientierung behandelt haben) Weiß ich, wieso die Singleton-Vorlage zur Prozeduralen Java-Programmierung so ist, wie sie ist?
  \item \emph{Java:} Kann ich je nach Zustand den Programmfluss ändern?
  \item \emph{Java:} Weiß ich, welche Variablen wo gespeichert werden?
  \item \emph{Java:} Kann ich in Java Prozeduren komplett schreiben?
  \item \emph{Java:} Weiß ich, für was die einzelnen Schlüsselworte in einer Signatur stehen?
\end{checklist}

\section{Blatt 9}

\begin{checklist}
  \item \emph{Java:} Kann ich mit Arrays umgehen?
  \item \emph{Java:} Kann ich eigene Verbunddatentypen definieren?
  \item \emph{Java:} (Nachdem wir jetzt auch Objektorientierung behandelt haben) Weiß ich, was es mit Verbunddatentypen auf sich hat?
  \item \emph{Java:} Weiß ich, welche Fehler wann auftreten können?
  \item \emph{Java:} Weiß ich, welche Variablen wann und wie "`zur Verfügung stehen"'?
  \item Weiß ich, was es mit der O-Notation auf sich hat?
\end{checklist}

Freiwillige Zusatzaufgaben: 2c, 7a, 7b, 7c, 7d

\section{Blatt 10}

\begin{checklist}
  \item \emph{Java:} Kann ich eine rekursive Prozedur in eine iterative übersetzen?
  \item \emph{Java:} Kann ich mit rekursiven Verbundtypen umgehen?
  \item \emph{Java:} Weiß ich, wohin Referenzen zeigen?
  \item \emph{Java:} Kenne ich den unterschied zwischen Call-by-Value und Call-by-Reference?
  \item Kann ich von einer gegeben Prozedur die Aufwandsfunktion bestimmen bzw. die Komplexitätsklasse angeben?
  \item \emph{Java:} Weiß ich, wann und wo Speicher belegt ist?
\end{checklist}

\section{Blatt 11}

\begin{checklist}
  \item Kenne ich das UML-Subset der Vorlesung?
\end{checklist}

\section{Blatt 12}

\begin{checklist}
  \item \emph{Java:} Kann ich ein UML-Klassendiagramm in Javacode umsetzen und umgekehrt?
  \item \emph{Java:} Wie kann ich darüber bestimmen, welche Instanzvariablen eines Objekts erreichbar sind?
  \item \emph{Java:} Weiß ich, welche Methoden static und welche nicht-static sein sollen?
\end{checklist}

\section{Blatt 13}

\begin{checklist}
  \item \emph{Java:} Weiß ich, wie ich durch das Typsystem navigieren kann?
  \item \emph{Java:} Kenne ich den Unterschied zwischen implements und extends?
  \item \emph{Java:} Weiß ich, welcher Code ausgeführt wird, wenn ich eine Methode auf einem Objekt aufrufe?
  \item Habe ich die Vorlesungsumfrage ausgefüllt?
\end{checklist}

\section{Blatt 14}

\begin{checklist}
  \item \emph{Java:} Kann ich mit Typ-Parametern umgehen?
  \item \emph{Java:} Kenne ich wichtige Collection-Klasen?
\end{checklist}

\end{document}