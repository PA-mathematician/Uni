\documentclass{article}

\usepackage[utf8x]{inputenc}
\usepackage[ngerman]{babel}
\usepackage{amsmath,amsfonts,amssymb}


\title{VerSy - Glossar}
\author{Manuel Hoffmann}

\begin{document}

\begin{itemize}
	\item \emph{Abtastung}\\
		Der Vorgang bei Scannern, wo es um dinge geht.

	\item \emph{Aktuator}\\
		TODO

	\item \emph{Babbeling Idiot Fehler}\\
		TODO

	\item \emph{Beacon Frame}\\
		TODO

	\item \emph{Baum}\\
		TODO

	\item \emph{Best Effort}\\
		TODO

	\item \emph{Bitfehler}\\
		TODO

	\item \emph{Bitmonitoring}\\
		TODO

	\item \emph{Bitstuffing}\\
		TODO

	\item \emph{Bittakt}\\
		TODO

	\item \emph{Blindzeit}\\
		Zeit, in der nicht gesendet und empfangen werden kann.

	\item \emph{Break-by-Wire}\\
		Bremsen ohne Hydraulik oder Pneumatik zwischen Bremspedal und Radbremse

	\item \emph{Bus Guardian}\\
		TODO

	\item \emph{Clear Channel Assessment}\\
		Bauteil, das eintreffende Energie aufsummiert. Kurze Aktion, 8 Symbole

	\item \emph{Clock Offset}\\
		Abweichung der Zeit von Uhren

	\item \emph{Clock Skew}\\
		Abweichungen der Geschwindigkeit von Uhren

	\item \emph{Closed Loop Control}\\
		Siehe \emph{Regelung}

	\item \emph{Contention Window}\\
		Wlan: Fenster aus dem Zufallswert gezogen wird\\
		Zigbee: Testzähler, ob das Medium wirklich frei ist. Wert >= 2, damit garantiert kein ACK überschrieben wird. Durch unterschiedliche Werte für das Contention Window könnte so auch Priorisierung einzelner Knoten vorgenommen werden -> Überprüfen.

	\item \emph{Dateneinheit}\\
		TODO
	
	\item \emph{Darstellungsdimension}\\
		TODO

	\item \emph{Dienstgüte}\\
		TODO

	\item \emph{Diskretes Medium}\\
		Siehe \emph{Zeitunabhängiges Medium}

	\item \emph{Dominanter Pegel}\\
		TODO

	\item \emph{Echtzeit}\\
		TODO

	\item \emph{Eingebettetes System}\\
		TODO

	\item \emph{Ereignisgetrigerte Kommunikation}\\
		Nachrichten werden gesendet, wenn entsprechendes Ereignis eintritt.

	\item \emph{Feedback Control}\\
		Siehe \emph{Regelung}

	\item \emph{Feldebene}\\
		TODO

	\item \emph{FlexRay}\\
		Echtzeitfähiger Bus

	\item \emph{Fragment Burst}\\
		TODO

	\item \emph{Füllverzögerung}\\
		TODO

	\item \emph{GTS Deskriptor}\\
		TODO

	\item \emph{Infrastruktur}\\
		TODO

	\item \emph{Interaktivität}\\
		TODO

	\item \emph{Jitter}\\
		Siehe \emph{Verzögerungsschwankung}

	\item \emph{Kommunikationsmedium}\\
		TODO

	\item \emph{Kommunikationszyklen}\\
		Aufteilung der Sendezeit in Zeitabschnitte fester Dauer bei FlexRay
	
	\item \emph{Kontinuierliches Medium}\\
		Siehe \emph{Zeitabhängiges Medium}

	\item \emph{Latency}\\
		Siehe \emph{Übertragungsverzögerung}

	\item \emph{Linienbus}\\
		TODO

	\item \emph{MAC-Rahmen}\\
		TODO

	\item \emph{Medium}\\
		Def.: Mittel zur Verbreitung und Darstellung von Informationen

	\item \emph{Mesh}\\
		TODO

	\item \emph{Mobilität}\\
		TODO

	\item \emph{Multimedia-System}\\
		Ein Multimedia-System im engeren Sinn ist durch die rechnergesteuerte, integrierte Erzeugung, Manipulation, Darstellung, Speicherung und Kommunikation von unabhängigen Informationen gekennzeichnet, die in mindestens einem kontinuierlichen (zeitabhängigen) und einem diskreten (zeitunabhängigen) Medium kodiert sind.

	\item \emph{Nutzdatenrate}\\
		TODO
	
	\item \emph{Open Loop Control}\\
		Siehe \emph{Steuerung}

	\item \emph{Perzeptionsmedium}\\
		Ein durch den Menschen wahrnehmbares Medium.

	\item \emph{Point Coordinator}\\
		TODO

	\item \emph{Polling Liste}\\
		TODO

	\item \emph{Präsentationsmedium}\\
		Medium zur Übergabe an der Schnittstelle

	\item \emph{Prozessleitebene}\\
		TODO

	\item \emph{Prüfsummenfehler}\\
		TODO

	\item \emph{Rahmen}\\
		TODO

	\item \emph{Regelung}\\
		TODO

	\item \emph{Repräsentationsmedium}\\
		Medium zur inneren Darstellung/Kodierung
	
	\item \emph{Rezessiver Pegel}\\
		TODO

	\item \emph{Schwach Periodisch}\\
		TODO

	\item \emph{Sensor}\\
		TODO

	\item \emph{Signalniveau}\\
		Wie viele bit kann ich pro Symbol codieren.

	\item \emph{Single-Hop-Bereich}\\
		Netzwerk, in dem jeder Knoten jeden anderen Knoten direkt, also ohne zwischenschritte über dritte Knoten erreichen kann.

	\item \emph{Slotted CSMA}\\
		Energiesparen durch abhören im Slottakt, nicht durchgehend.

	\item \emph{Speichermedium}\\
		TODO

	\item \emph{Sporadisch}\\
		TODO

	\item \emph{Steer-by-Wire}\\
		Lenken ohne mechanische Verbindung zwischen Lenkrad und gelenkten Rädern
	
	\item \emph{Stern}\\
		TODO

	\item \emph{Steuerung}\\
		TODO

	\item \emph{Streng Periodisch}\\
		TODO

	\item \emph{Superframe}\\
		TODO

	\item \emph{Symbol}\\
		TODO

	\item \emph{Sync Byte}\\
		Bitmuster 01010101 zur Bittaktsynchronisation.

	\item \emph{Sync Break}\\
		Bitmuster mit $\geq 13$ dominanten Bits (LIN); verletzt absichtlich Default-Zeichenformat

	\item \emph{Synchronisationsfehler}\\
		TODO

	\item \emph{Throughput -> Übertragungsrate}\\
		TODO

	\item \emph{Topologie}\\
		TODO

	\item \emph{Übertragungseinheit}\\
		TODO

	\item \emph{Übertragungsrate}\\
		TODO

	\item \emph{Übertragungsverzögerung}\\
		TODO

	\item \emph{Verfälschungsrate}\\
		TODO

	\item \emph{Verlustrate}\\
		TODO

	\item \emph{Verzögerungsschwankung}\\
		TODO

	\item \emph{Zellebene}\\
		TODO

	\item \emph{Zeitabhängiges Medium}\\
		Medium mit zeitbezogener Information wie Musik, Sprache, Video, Sensordaten von Mäusen
	
	\item \emph{Zeitliche Kapselung}\\
		TODO

	\item \emph{Zeitgetriggerte Kommunikation}\\
		Nachrichten werden zu vorherbestimmten Zeitpunkten gesendet.

	\item \emph{Zeitunabhängiges Medium}\\
		Medium ohne zeitbezogene Information wie Text, Fotos

	\item \emph{Zwischenankunftsintervall}\\
		TODO
\end{itemize}

\end{document}